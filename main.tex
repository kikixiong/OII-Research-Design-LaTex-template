\documentclass[12pt, letterpaper]{article} % Set 12pt font size and US letter paper size (change to a4paper if needed)

% =======================================================
% PACKAGES (PREAMBLE)
% =======================================================

\usepackage[T1]{fontenc} 
\usepackage{newtxtext,newtxmath} 
\usepackage[margin=1in]{geometry}
\usepackage{setspace}
\usepackage{natbib} 
\usepackage{caption} 
\usepackage{booktabs} 

% =======================================================
% FORMATTING SETUP
% =======================================================

% Set double line spacing. 
% This command is placed here to apply to the main text. 
% Note: The setspace package is designed to keep captions, footnotes, 
% and list environments single-spaced, which is a common academic standard.
\doublespacing 

% Set a default font size for captions and small elements to 12pt to meet the requirement 
% "Do not use less than 12 points, including in any tables and figures." 
% WARNING: This overrides the standard LaTeX behavior and might look unusual 
% for footnotes/captions, but it strictly follows the rule.
\captionsetup{font=normalsize} % normalsize for 12pt document is 12pt
\renewcommand{\footnotesize}{\normalsize} % Ensures footnotes are also 12pt (normalsize)

% =======================================================
% DOCUMENT START
% =======================================================
\begin{document}

\title{Template for OII Research Design}
\author{Your Name / Student ID}
\date{October 2025}

\maketitle 

% \begin{abstract}
% This document serves as a LaTeX template to comply with specific academic formatting rules: double-spaced, 1-inch margins, and 12-point serif font throughout. The template utilizes the \texttt{geometry}, \texttt{setspace}, and \texttt{newtxtext} packages to enforce these strict requirements. All elements, including the abstract, body text, figures, tables, and bibliography, are set to 12 points or greater.
% \end{abstract}

\section{Summary of the research}
This template ensures strict adherence to all mandatory formatting rules, including 12-point \texttt{serif} font, a 1-inch margin on all sides, and full double-spacing \citep{smith2020example}. The combination of \texttt{geometry} and \texttt{setspace} packages controls the layout precisely. We have ensured that this text remains fully double-spaced.

\section{Methodology and Results}

\subsection{Table Example}
We must ensure that all elements within tables and figures also use a font size no smaller than 12pt, as demonstrated in Table \ref{tab:example}.

\begin{table}[h]
    \centering
    \caption{Example Table with 12pt Serif Font}
    \label{tab:example}
    \begin{tabular}{l c c}
        \toprule
        \textbf{Metric (12pt)} & \textbf{Value A (12pt)} & \textbf{Value B (12pt)} \\
        \midrule
        Observation 1 & 1.234 & 5.678 \\
        Observation 2 & 9.012 & 3.456 \\
        \bottomrule
    \end{tabular}
\end{table}

\subsection{Figure Example}
Figure captions must also be 12pt. The \texttt{newtxtext} package provides the appropriate serif typeface \citep[see also][]{jones2022another}.

\begin{figure}[h]
    \centering
    % Placeholder for a graphic. Replace \rule with \includegraphics{your-image.pdf}
    \rule{0.6\textwidth}{4cm} 
    \caption{Example Figure with a required 12pt caption font size.}
    \label{fig:figure_example}
\end{figure}

\section{Conclusion}
The template successfully implements all specified formatting requirements.

\newpage % Start the bibliography on a new page

% =======================================================
% BIBLIOGRAPHY
% =======================================================
% Use plainnat as a common bibliography style.
\bibliographystyle{plainnat} 
\bibliography{references} 

\end{document}